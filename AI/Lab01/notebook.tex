
% Default to the notebook output style

    


% Inherit from the specified cell style.




    
\documentclass[11pt]{article}

    
    
    \usepackage[T1]{fontenc}
    % Nicer default font (+ math font) than Computer Modern for most use cases
    \usepackage{mathpazo}

    % Basic figure setup, for now with no caption control since it's done
    % automatically by Pandoc (which extracts ![](path) syntax from Markdown).
    \usepackage{graphicx}
    % We will generate all images so they have a width \maxwidth. This means
    % that they will get their normal width if they fit onto the page, but
    % are scaled down if they would overflow the margins.
    \makeatletter
    \def\maxwidth{\ifdim\Gin@nat@width>\linewidth\linewidth
    \else\Gin@nat@width\fi}
    \makeatother
    \let\Oldincludegraphics\includegraphics
    % Set max figure width to be 80% of text width, for now hardcoded.
    \renewcommand{\includegraphics}[1]{\Oldincludegraphics[width=.8\maxwidth]{#1}}
    % Ensure that by default, figures have no caption (until we provide a
    % proper Figure object with a Caption API and a way to capture that
    % in the conversion process - todo).
    \usepackage{caption}
    \DeclareCaptionLabelFormat{nolabel}{}
    \captionsetup{labelformat=nolabel}

    \usepackage{adjustbox} % Used to constrain images to a maximum size 
    \usepackage{xcolor} % Allow colors to be defined
    \usepackage{enumerate} % Needed for markdown enumerations to work
    \usepackage{geometry} % Used to adjust the document margins
    \usepackage{amsmath} % Equations
    \usepackage{amssymb} % Equations
    \usepackage{textcomp} % defines textquotesingle
    % Hack from http://tex.stackexchange.com/a/47451/13684:
    \AtBeginDocument{%
        \def\PYZsq{\textquotesingle}% Upright quotes in Pygmentized code
    }
    \usepackage{upquote} % Upright quotes for verbatim code
    \usepackage{eurosym} % defines \euro
    \usepackage[mathletters]{ucs} % Extended unicode (utf-8) support
    \usepackage[utf8x]{inputenc} % Allow utf-8 characters in the tex document
    \usepackage{fancyvrb} % verbatim replacement that allows latex
    \usepackage{grffile} % extends the file name processing of package graphics 
                         % to support a larger range 
    % The hyperref package gives us a pdf with properly built
    % internal navigation ('pdf bookmarks' for the table of contents,
    % internal cross-reference links, web links for URLs, etc.)
    \usepackage{hyperref}
    \usepackage{longtable} % longtable support required by pandoc >1.10
    \usepackage{booktabs}  % table support for pandoc > 1.12.2
    \usepackage[inline]{enumitem} % IRkernel/repr support (it uses the enumerate* environment)
    \usepackage[normalem]{ulem} % ulem is needed to support strikethroughs (\sout)
                                % normalem makes italics be italics, not underlines
    

    
    
    % Colors for the hyperref package
    \definecolor{urlcolor}{rgb}{0,.145,.698}
    \definecolor{linkcolor}{rgb}{.71,0.21,0.01}
    \definecolor{citecolor}{rgb}{.12,.54,.11}

    % ANSI colors
    \definecolor{ansi-black}{HTML}{3E424D}
    \definecolor{ansi-black-intense}{HTML}{282C36}
    \definecolor{ansi-red}{HTML}{E75C58}
    \definecolor{ansi-red-intense}{HTML}{B22B31}
    \definecolor{ansi-green}{HTML}{00A250}
    \definecolor{ansi-green-intense}{HTML}{007427}
    \definecolor{ansi-yellow}{HTML}{DDB62B}
    \definecolor{ansi-yellow-intense}{HTML}{B27D12}
    \definecolor{ansi-blue}{HTML}{208FFB}
    \definecolor{ansi-blue-intense}{HTML}{0065CA}
    \definecolor{ansi-magenta}{HTML}{D160C4}
    \definecolor{ansi-magenta-intense}{HTML}{A03196}
    \definecolor{ansi-cyan}{HTML}{60C6C8}
    \definecolor{ansi-cyan-intense}{HTML}{258F8F}
    \definecolor{ansi-white}{HTML}{C5C1B4}
    \definecolor{ansi-white-intense}{HTML}{A1A6B2}

    % commands and environments needed by pandoc snippets
    % extracted from the output of `pandoc -s`
    \providecommand{\tightlist}{%
      \setlength{\itemsep}{0pt}\setlength{\parskip}{0pt}}
    \DefineVerbatimEnvironment{Highlighting}{Verbatim}{commandchars=\\\{\}}
    % Add ',fontsize=\small' for more characters per line
    \newenvironment{Shaded}{}{}
    \newcommand{\KeywordTok}[1]{\textcolor[rgb]{0.00,0.44,0.13}{\textbf{{#1}}}}
    \newcommand{\DataTypeTok}[1]{\textcolor[rgb]{0.56,0.13,0.00}{{#1}}}
    \newcommand{\DecValTok}[1]{\textcolor[rgb]{0.25,0.63,0.44}{{#1}}}
    \newcommand{\BaseNTok}[1]{\textcolor[rgb]{0.25,0.63,0.44}{{#1}}}
    \newcommand{\FloatTok}[1]{\textcolor[rgb]{0.25,0.63,0.44}{{#1}}}
    \newcommand{\CharTok}[1]{\textcolor[rgb]{0.25,0.44,0.63}{{#1}}}
    \newcommand{\StringTok}[1]{\textcolor[rgb]{0.25,0.44,0.63}{{#1}}}
    \newcommand{\CommentTok}[1]{\textcolor[rgb]{0.38,0.63,0.69}{\textit{{#1}}}}
    \newcommand{\OtherTok}[1]{\textcolor[rgb]{0.00,0.44,0.13}{{#1}}}
    \newcommand{\AlertTok}[1]{\textcolor[rgb]{1.00,0.00,0.00}{\textbf{{#1}}}}
    \newcommand{\FunctionTok}[1]{\textcolor[rgb]{0.02,0.16,0.49}{{#1}}}
    \newcommand{\RegionMarkerTok}[1]{{#1}}
    \newcommand{\ErrorTok}[1]{\textcolor[rgb]{1.00,0.00,0.00}{\textbf{{#1}}}}
    \newcommand{\NormalTok}[1]{{#1}}
    
    % Additional commands for more recent versions of Pandoc
    \newcommand{\ConstantTok}[1]{\textcolor[rgb]{0.53,0.00,0.00}{{#1}}}
    \newcommand{\SpecialCharTok}[1]{\textcolor[rgb]{0.25,0.44,0.63}{{#1}}}
    \newcommand{\VerbatimStringTok}[1]{\textcolor[rgb]{0.25,0.44,0.63}{{#1}}}
    \newcommand{\SpecialStringTok}[1]{\textcolor[rgb]{0.73,0.40,0.53}{{#1}}}
    \newcommand{\ImportTok}[1]{{#1}}
    \newcommand{\DocumentationTok}[1]{\textcolor[rgb]{0.73,0.13,0.13}{\textit{{#1}}}}
    \newcommand{\AnnotationTok}[1]{\textcolor[rgb]{0.38,0.63,0.69}{\textbf{\textit{{#1}}}}}
    \newcommand{\CommentVarTok}[1]{\textcolor[rgb]{0.38,0.63,0.69}{\textbf{\textit{{#1}}}}}
    \newcommand{\VariableTok}[1]{\textcolor[rgb]{0.10,0.09,0.49}{{#1}}}
    \newcommand{\ControlFlowTok}[1]{\textcolor[rgb]{0.00,0.44,0.13}{\textbf{{#1}}}}
    \newcommand{\OperatorTok}[1]{\textcolor[rgb]{0.40,0.40,0.40}{{#1}}}
    \newcommand{\BuiltInTok}[1]{{#1}}
    \newcommand{\ExtensionTok}[1]{{#1}}
    \newcommand{\PreprocessorTok}[1]{\textcolor[rgb]{0.74,0.48,0.00}{{#1}}}
    \newcommand{\AttributeTok}[1]{\textcolor[rgb]{0.49,0.56,0.16}{{#1}}}
    \newcommand{\InformationTok}[1]{\textcolor[rgb]{0.38,0.63,0.69}{\textbf{\textit{{#1}}}}}
    \newcommand{\WarningTok}[1]{\textcolor[rgb]{0.38,0.63,0.69}{\textbf{\textit{{#1}}}}}
    
    
    % Define a nice break command that doesn't care if a line doesn't already
    % exist.
    \def\br{\hspace*{\fill} \\* }
    % Math Jax compatability definitions
    \def\gt{>}
    \def\lt{<}
    % Document parameters
    \title{lab01\_notebook}
    
    
    

    % Pygments definitions
    
\makeatletter
\def\PY@reset{\let\PY@it=\relax \let\PY@bf=\relax%
    \let\PY@ul=\relax \let\PY@tc=\relax%
    \let\PY@bc=\relax \let\PY@ff=\relax}
\def\PY@tok#1{\csname PY@tok@#1\endcsname}
\def\PY@toks#1+{\ifx\relax#1\empty\else%
    \PY@tok{#1}\expandafter\PY@toks\fi}
\def\PY@do#1{\PY@bc{\PY@tc{\PY@ul{%
    \PY@it{\PY@bf{\PY@ff{#1}}}}}}}
\def\PY#1#2{\PY@reset\PY@toks#1+\relax+\PY@do{#2}}

\expandafter\def\csname PY@tok@w\endcsname{\def\PY@tc##1{\textcolor[rgb]{0.73,0.73,0.73}{##1}}}
\expandafter\def\csname PY@tok@c\endcsname{\let\PY@it=\textit\def\PY@tc##1{\textcolor[rgb]{0.25,0.50,0.50}{##1}}}
\expandafter\def\csname PY@tok@cp\endcsname{\def\PY@tc##1{\textcolor[rgb]{0.74,0.48,0.00}{##1}}}
\expandafter\def\csname PY@tok@k\endcsname{\let\PY@bf=\textbf\def\PY@tc##1{\textcolor[rgb]{0.00,0.50,0.00}{##1}}}
\expandafter\def\csname PY@tok@kp\endcsname{\def\PY@tc##1{\textcolor[rgb]{0.00,0.50,0.00}{##1}}}
\expandafter\def\csname PY@tok@kt\endcsname{\def\PY@tc##1{\textcolor[rgb]{0.69,0.00,0.25}{##1}}}
\expandafter\def\csname PY@tok@o\endcsname{\def\PY@tc##1{\textcolor[rgb]{0.40,0.40,0.40}{##1}}}
\expandafter\def\csname PY@tok@ow\endcsname{\let\PY@bf=\textbf\def\PY@tc##1{\textcolor[rgb]{0.67,0.13,1.00}{##1}}}
\expandafter\def\csname PY@tok@nb\endcsname{\def\PY@tc##1{\textcolor[rgb]{0.00,0.50,0.00}{##1}}}
\expandafter\def\csname PY@tok@nf\endcsname{\def\PY@tc##1{\textcolor[rgb]{0.00,0.00,1.00}{##1}}}
\expandafter\def\csname PY@tok@nc\endcsname{\let\PY@bf=\textbf\def\PY@tc##1{\textcolor[rgb]{0.00,0.00,1.00}{##1}}}
\expandafter\def\csname PY@tok@nn\endcsname{\let\PY@bf=\textbf\def\PY@tc##1{\textcolor[rgb]{0.00,0.00,1.00}{##1}}}
\expandafter\def\csname PY@tok@ne\endcsname{\let\PY@bf=\textbf\def\PY@tc##1{\textcolor[rgb]{0.82,0.25,0.23}{##1}}}
\expandafter\def\csname PY@tok@nv\endcsname{\def\PY@tc##1{\textcolor[rgb]{0.10,0.09,0.49}{##1}}}
\expandafter\def\csname PY@tok@no\endcsname{\def\PY@tc##1{\textcolor[rgb]{0.53,0.00,0.00}{##1}}}
\expandafter\def\csname PY@tok@nl\endcsname{\def\PY@tc##1{\textcolor[rgb]{0.63,0.63,0.00}{##1}}}
\expandafter\def\csname PY@tok@ni\endcsname{\let\PY@bf=\textbf\def\PY@tc##1{\textcolor[rgb]{0.60,0.60,0.60}{##1}}}
\expandafter\def\csname PY@tok@na\endcsname{\def\PY@tc##1{\textcolor[rgb]{0.49,0.56,0.16}{##1}}}
\expandafter\def\csname PY@tok@nt\endcsname{\let\PY@bf=\textbf\def\PY@tc##1{\textcolor[rgb]{0.00,0.50,0.00}{##1}}}
\expandafter\def\csname PY@tok@nd\endcsname{\def\PY@tc##1{\textcolor[rgb]{0.67,0.13,1.00}{##1}}}
\expandafter\def\csname PY@tok@s\endcsname{\def\PY@tc##1{\textcolor[rgb]{0.73,0.13,0.13}{##1}}}
\expandafter\def\csname PY@tok@sd\endcsname{\let\PY@it=\textit\def\PY@tc##1{\textcolor[rgb]{0.73,0.13,0.13}{##1}}}
\expandafter\def\csname PY@tok@si\endcsname{\let\PY@bf=\textbf\def\PY@tc##1{\textcolor[rgb]{0.73,0.40,0.53}{##1}}}
\expandafter\def\csname PY@tok@se\endcsname{\let\PY@bf=\textbf\def\PY@tc##1{\textcolor[rgb]{0.73,0.40,0.13}{##1}}}
\expandafter\def\csname PY@tok@sr\endcsname{\def\PY@tc##1{\textcolor[rgb]{0.73,0.40,0.53}{##1}}}
\expandafter\def\csname PY@tok@ss\endcsname{\def\PY@tc##1{\textcolor[rgb]{0.10,0.09,0.49}{##1}}}
\expandafter\def\csname PY@tok@sx\endcsname{\def\PY@tc##1{\textcolor[rgb]{0.00,0.50,0.00}{##1}}}
\expandafter\def\csname PY@tok@m\endcsname{\def\PY@tc##1{\textcolor[rgb]{0.40,0.40,0.40}{##1}}}
\expandafter\def\csname PY@tok@gh\endcsname{\let\PY@bf=\textbf\def\PY@tc##1{\textcolor[rgb]{0.00,0.00,0.50}{##1}}}
\expandafter\def\csname PY@tok@gu\endcsname{\let\PY@bf=\textbf\def\PY@tc##1{\textcolor[rgb]{0.50,0.00,0.50}{##1}}}
\expandafter\def\csname PY@tok@gd\endcsname{\def\PY@tc##1{\textcolor[rgb]{0.63,0.00,0.00}{##1}}}
\expandafter\def\csname PY@tok@gi\endcsname{\def\PY@tc##1{\textcolor[rgb]{0.00,0.63,0.00}{##1}}}
\expandafter\def\csname PY@tok@gr\endcsname{\def\PY@tc##1{\textcolor[rgb]{1.00,0.00,0.00}{##1}}}
\expandafter\def\csname PY@tok@ge\endcsname{\let\PY@it=\textit}
\expandafter\def\csname PY@tok@gs\endcsname{\let\PY@bf=\textbf}
\expandafter\def\csname PY@tok@gp\endcsname{\let\PY@bf=\textbf\def\PY@tc##1{\textcolor[rgb]{0.00,0.00,0.50}{##1}}}
\expandafter\def\csname PY@tok@go\endcsname{\def\PY@tc##1{\textcolor[rgb]{0.53,0.53,0.53}{##1}}}
\expandafter\def\csname PY@tok@gt\endcsname{\def\PY@tc##1{\textcolor[rgb]{0.00,0.27,0.87}{##1}}}
\expandafter\def\csname PY@tok@err\endcsname{\def\PY@bc##1{\setlength{\fboxsep}{0pt}\fcolorbox[rgb]{1.00,0.00,0.00}{1,1,1}{\strut ##1}}}
\expandafter\def\csname PY@tok@kc\endcsname{\let\PY@bf=\textbf\def\PY@tc##1{\textcolor[rgb]{0.00,0.50,0.00}{##1}}}
\expandafter\def\csname PY@tok@kd\endcsname{\let\PY@bf=\textbf\def\PY@tc##1{\textcolor[rgb]{0.00,0.50,0.00}{##1}}}
\expandafter\def\csname PY@tok@kn\endcsname{\let\PY@bf=\textbf\def\PY@tc##1{\textcolor[rgb]{0.00,0.50,0.00}{##1}}}
\expandafter\def\csname PY@tok@kr\endcsname{\let\PY@bf=\textbf\def\PY@tc##1{\textcolor[rgb]{0.00,0.50,0.00}{##1}}}
\expandafter\def\csname PY@tok@bp\endcsname{\def\PY@tc##1{\textcolor[rgb]{0.00,0.50,0.00}{##1}}}
\expandafter\def\csname PY@tok@fm\endcsname{\def\PY@tc##1{\textcolor[rgb]{0.00,0.00,1.00}{##1}}}
\expandafter\def\csname PY@tok@vc\endcsname{\def\PY@tc##1{\textcolor[rgb]{0.10,0.09,0.49}{##1}}}
\expandafter\def\csname PY@tok@vg\endcsname{\def\PY@tc##1{\textcolor[rgb]{0.10,0.09,0.49}{##1}}}
\expandafter\def\csname PY@tok@vi\endcsname{\def\PY@tc##1{\textcolor[rgb]{0.10,0.09,0.49}{##1}}}
\expandafter\def\csname PY@tok@vm\endcsname{\def\PY@tc##1{\textcolor[rgb]{0.10,0.09,0.49}{##1}}}
\expandafter\def\csname PY@tok@sa\endcsname{\def\PY@tc##1{\textcolor[rgb]{0.73,0.13,0.13}{##1}}}
\expandafter\def\csname PY@tok@sb\endcsname{\def\PY@tc##1{\textcolor[rgb]{0.73,0.13,0.13}{##1}}}
\expandafter\def\csname PY@tok@sc\endcsname{\def\PY@tc##1{\textcolor[rgb]{0.73,0.13,0.13}{##1}}}
\expandafter\def\csname PY@tok@dl\endcsname{\def\PY@tc##1{\textcolor[rgb]{0.73,0.13,0.13}{##1}}}
\expandafter\def\csname PY@tok@s2\endcsname{\def\PY@tc##1{\textcolor[rgb]{0.73,0.13,0.13}{##1}}}
\expandafter\def\csname PY@tok@sh\endcsname{\def\PY@tc##1{\textcolor[rgb]{0.73,0.13,0.13}{##1}}}
\expandafter\def\csname PY@tok@s1\endcsname{\def\PY@tc##1{\textcolor[rgb]{0.73,0.13,0.13}{##1}}}
\expandafter\def\csname PY@tok@mb\endcsname{\def\PY@tc##1{\textcolor[rgb]{0.40,0.40,0.40}{##1}}}
\expandafter\def\csname PY@tok@mf\endcsname{\def\PY@tc##1{\textcolor[rgb]{0.40,0.40,0.40}{##1}}}
\expandafter\def\csname PY@tok@mh\endcsname{\def\PY@tc##1{\textcolor[rgb]{0.40,0.40,0.40}{##1}}}
\expandafter\def\csname PY@tok@mi\endcsname{\def\PY@tc##1{\textcolor[rgb]{0.40,0.40,0.40}{##1}}}
\expandafter\def\csname PY@tok@il\endcsname{\def\PY@tc##1{\textcolor[rgb]{0.40,0.40,0.40}{##1}}}
\expandafter\def\csname PY@tok@mo\endcsname{\def\PY@tc##1{\textcolor[rgb]{0.40,0.40,0.40}{##1}}}
\expandafter\def\csname PY@tok@ch\endcsname{\let\PY@it=\textit\def\PY@tc##1{\textcolor[rgb]{0.25,0.50,0.50}{##1}}}
\expandafter\def\csname PY@tok@cm\endcsname{\let\PY@it=\textit\def\PY@tc##1{\textcolor[rgb]{0.25,0.50,0.50}{##1}}}
\expandafter\def\csname PY@tok@cpf\endcsname{\let\PY@it=\textit\def\PY@tc##1{\textcolor[rgb]{0.25,0.50,0.50}{##1}}}
\expandafter\def\csname PY@tok@c1\endcsname{\let\PY@it=\textit\def\PY@tc##1{\textcolor[rgb]{0.25,0.50,0.50}{##1}}}
\expandafter\def\csname PY@tok@cs\endcsname{\let\PY@it=\textit\def\PY@tc##1{\textcolor[rgb]{0.25,0.50,0.50}{##1}}}

\def\PYZbs{\char`\\}
\def\PYZus{\char`\_}
\def\PYZob{\char`\{}
\def\PYZcb{\char`\}}
\def\PYZca{\char`\^}
\def\PYZam{\char`\&}
\def\PYZlt{\char`\<}
\def\PYZgt{\char`\>}
\def\PYZsh{\char`\#}
\def\PYZpc{\char`\%}
\def\PYZdl{\char`\$}
\def\PYZhy{\char`\-}
\def\PYZsq{\char`\'}
\def\PYZdq{\char`\"}
\def\PYZti{\char`\~}
% for compatibility with earlier versions
\def\PYZat{@}
\def\PYZlb{[}
\def\PYZrb{]}
\makeatother


    % Exact colors from NB
    \definecolor{incolor}{rgb}{0.0, 0.0, 0.5}
    \definecolor{outcolor}{rgb}{0.545, 0.0, 0.0}



    
    % Prevent overflowing lines due to hard-to-break entities
    \sloppy 
    % Setup hyperref package
    \hypersetup{
      breaklinks=true,  % so long urls are correctly broken across lines
      colorlinks=true,
      urlcolor=urlcolor,
      linkcolor=linkcolor,
      citecolor=citecolor,
      }
    % Slightly bigger margins than the latex defaults
    
    \geometry{verbose,tmargin=1in,bmargin=1in,lmargin=1in,rmargin=1in}
    
    

    \begin{document}
    
    
    \maketitle
    
    

    
    \subsection{3. Exploring data tables with
Pandas}\label{exploring-data-tables-with-pandas}

\begin{enumerate}
\def\labelenumi{\arabic{enumi}.}
\tightlist
\item
  Use Pandas to read the house prices data. How many columns and rows
  are there in this dataset?
\item
  The first step I usually do is to use commands like pandas.head() to
  print a few rows of data. Look around what kind of features are
  available and read data description.txt for more info. Try to
  understand as much as you can. Pick three features you think will be
  good predictors of house prices and explain what they are.
\item
  How many unique conditions are there in SaleCondition? Use Pandas to
  find out how many samples are labeled with each condition. What do you
  learn from doing this?
\item
  Select one variable you picked in b., do you want to know something
  more about that variable? Use Pandas to answer your own question and
  de- scribe what you did shortly here.
\end{enumerate}

    \begin{Verbatim}[commandchars=\\\{\}]
{\color{incolor}In [{\color{incolor}1}]:} \PY{c+c1}{\PYZsh{}Import Libraries}
        \PY{k+kn}{import} \PY{n+nn}{pandas} \PY{k}{as} \PY{n+nn}{pd}
        \PY{k+kn}{import} \PY{n+nn}{numpy} \PY{k}{as} \PY{n+nn}{np}
        \PY{k+kn}{import} \PY{n+nn}{seaborn} \PY{k}{as} \PY{n+nn}{sns}
        \PY{k+kn}{from} \PY{n+nn}{sklearn} \PY{k}{import} \PY{n}{linear\PYZus{}model}
        \PY{k+kn}{from} \PY{n+nn}{sklearn}\PY{n+nn}{.}\PY{n+nn}{model\PYZus{}selection} \PY{k}{import} \PY{n}{cross\PYZus{}val\PYZus{}predict}
        \PY{k+kn}{from} \PY{n+nn}{scipy} \PY{k}{import} \PY{n}{stats}
        \PY{k+kn}{import} \PY{n+nn}{matplotlib}\PY{n+nn}{.}\PY{n+nn}{pyplot} \PY{k}{as} \PY{n+nn}{plt}
\end{Verbatim}


    \section{Answer 3.1}\label{answer-3.1}

read data using 'read\_csv' there are 1460 rows and 81 columns in data
set

    \begin{Verbatim}[commandchars=\\\{\}]
{\color{incolor}In [{\color{incolor}2}]:} \PY{n}{Location} \PY{o}{=} \PY{l+s+sa}{r}\PY{l+s+s1}{\PYZsq{}}\PY{l+s+s1}{C:}\PY{l+s+s1}{\PYZbs{}}\PY{l+s+s1}{Users}\PY{l+s+s1}{\PYZbs{}}\PY{l+s+s1}{dell}\PY{l+s+s1}{\PYZbs{}}\PY{l+s+s1}{Desktop}\PY{l+s+s1}{\PYZbs{}}\PY{l+s+s1}{module8\PYZam{}9}\PY{l+s+s1}{\PYZbs{}}\PY{l+s+s1}{AI}\PY{l+s+s1}{\PYZbs{}}\PY{l+s+s1}{Lab01}\PY{l+s+s1}{\PYZbs{}}\PY{l+s+s1}{train.csv}\PY{l+s+s1}{\PYZsq{}}
        \PY{n}{df} \PY{o}{=} \PY{n}{pd}\PY{o}{.}\PY{n}{read\PYZus{}csv}\PY{p}{(}\PY{n}{Location}\PY{p}{)}
\end{Verbatim}


    \section{Answer 3.2}\label{answer-3.2}

The Three features are 1. MSSubClass : Type of house 2. LotArea : Lot
size( Area of house) 3. LotFrontage : Area Front side of house

    \begin{Verbatim}[commandchars=\\\{\}]
{\color{incolor}In [{\color{incolor}3}]:} \PY{n}{df}\PY{o}{.}\PY{n}{head}\PY{p}{(}\PY{p}{)}
\end{Verbatim}


\begin{Verbatim}[commandchars=\\\{\}]
{\color{outcolor}Out[{\color{outcolor}3}]:}    Id  MSSubClass MSZoning  LotFrontage  LotArea Street Alley LotShape  \textbackslash{}
        0   1          60       RL         65.0     8450   Pave   NaN      Reg   
        1   2          20       RL         80.0     9600   Pave   NaN      Reg   
        2   3          60       RL         68.0    11250   Pave   NaN      IR1   
        3   4          70       RL         60.0     9550   Pave   NaN      IR1   
        4   5          60       RL         84.0    14260   Pave   NaN      IR1   
        
          LandContour Utilities    {\ldots}     PoolArea PoolQC Fence MiscFeature MiscVal  \textbackslash{}
        0         Lvl    AllPub    {\ldots}            0    NaN   NaN         NaN       0   
        1         Lvl    AllPub    {\ldots}            0    NaN   NaN         NaN       0   
        2         Lvl    AllPub    {\ldots}            0    NaN   NaN         NaN       0   
        3         Lvl    AllPub    {\ldots}            0    NaN   NaN         NaN       0   
        4         Lvl    AllPub    {\ldots}            0    NaN   NaN         NaN       0   
        
          MoSold YrSold  SaleType  SaleCondition  SalePrice  
        0      2   2008        WD         Normal     208500  
        1      5   2007        WD         Normal     181500  
        2      9   2008        WD         Normal     223500  
        3      2   2006        WD        Abnorml     140000  
        4     12   2008        WD         Normal     250000  
        
        [5 rows x 81 columns]
\end{Verbatim}
            
    \begin{Verbatim}[commandchars=\\\{\}]
{\color{incolor}In [{\color{incolor}4}]:} \PY{n}{df}\PY{o}{.}\PY{n}{dtypes} \PY{c+c1}{\PYZsh{}Check data types of all features}
\end{Verbatim}


\begin{Verbatim}[commandchars=\\\{\}]
{\color{outcolor}Out[{\color{outcolor}4}]:} Id                 int64
        MSSubClass         int64
        MSZoning          object
        LotFrontage      float64
        LotArea            int64
        Street            object
        Alley             object
        LotShape          object
        LandContour       object
        Utilities         object
        LotConfig         object
        LandSlope         object
        Neighborhood      object
        Condition1        object
        Condition2        object
        BldgType          object
        HouseStyle        object
        OverallQual        int64
        OverallCond        int64
        YearBuilt          int64
        YearRemodAdd       int64
        RoofStyle         object
        RoofMatl          object
        Exterior1st       object
        Exterior2nd       object
        MasVnrType        object
        MasVnrArea       float64
        ExterQual         object
        ExterCond         object
        Foundation        object
                          {\ldots}   
        BedroomAbvGr       int64
        KitchenAbvGr       int64
        KitchenQual       object
        TotRmsAbvGrd       int64
        Functional        object
        Fireplaces         int64
        FireplaceQu       object
        GarageType        object
        GarageYrBlt      float64
        GarageFinish      object
        GarageCars         int64
        GarageArea         int64
        GarageQual        object
        GarageCond        object
        PavedDrive        object
        WoodDeckSF         int64
        OpenPorchSF        int64
        EnclosedPorch      int64
        3SsnPorch          int64
        ScreenPorch        int64
        PoolArea           int64
        PoolQC            object
        Fence             object
        MiscFeature       object
        MiscVal            int64
        MoSold             int64
        YrSold             int64
        SaleType          object
        SaleCondition     object
        SalePrice          int64
        Length: 81, dtype: object
\end{Verbatim}
            
    \section{Answer 3.3}\label{answer-3.3}

There are 6 condition: 'Normal', 'Abnorml', 'Partial', 'AdjLand',
'Alloca', 'Family' 1. 'Normal' = 1,198 samples 2. 'Abnorml' = 101
samples 3. 'Partial' = 125 samples 4. 'AdjLand' = 4 samples 5. 'Alloca'
= 12 samples 6. 'Family' = 20 samples

I learned about how to find unique condition and how to prepare data to
find how many number of each condition

    \begin{Verbatim}[commandchars=\\\{\}]
{\color{incolor}In [{\color{incolor}5}]:} \PY{n}{df}\PY{p}{[}\PY{l+s+s1}{\PYZsq{}}\PY{l+s+s1}{SaleCondition}\PY{l+s+s1}{\PYZsq{}}\PY{p}{]}\PY{o}{.}\PY{n}{unique}\PY{p}{(}\PY{p}{)} \PY{c+c1}{\PYZsh{}find unique condition}
\end{Verbatim}


\begin{Verbatim}[commandchars=\\\{\}]
{\color{outcolor}Out[{\color{outcolor}5}]:} array(['Normal', 'Abnorml', 'Partial', 'AdjLand', 'Alloca', 'Family'],
              dtype=object)
\end{Verbatim}
            
    \begin{Verbatim}[commandchars=\\\{\}]
{\color{incolor}In [{\color{incolor}6}]:} \PY{c+c1}{\PYZsh{} find how many data in each condition}
        \PY{n+nb}{print}\PY{p}{(}\PY{n+nb}{len}\PY{p}{(}\PY{n}{df}\PY{p}{[}\PY{n}{df}\PY{p}{[}\PY{l+s+s1}{\PYZsq{}}\PY{l+s+s1}{SaleCondition}\PY{l+s+s1}{\PYZsq{}}\PY{p}{]} \PY{o}{==} \PY{l+s+s1}{\PYZsq{}}\PY{l+s+s1}{Normal}\PY{l+s+s1}{\PYZsq{}}\PY{p}{]}\PY{p}{)}\PY{p}{)}
        \PY{n+nb}{print}\PY{p}{(}\PY{n+nb}{len}\PY{p}{(}\PY{n}{df}\PY{p}{[}\PY{n}{df}\PY{p}{[}\PY{l+s+s1}{\PYZsq{}}\PY{l+s+s1}{SaleCondition}\PY{l+s+s1}{\PYZsq{}}\PY{p}{]} \PY{o}{==} \PY{l+s+s1}{\PYZsq{}}\PY{l+s+s1}{Abnorml}\PY{l+s+s1}{\PYZsq{}}\PY{p}{]}\PY{p}{)}\PY{p}{)}
        \PY{n+nb}{print}\PY{p}{(}\PY{n+nb}{len}\PY{p}{(}\PY{n}{df}\PY{p}{[}\PY{n}{df}\PY{p}{[}\PY{l+s+s1}{\PYZsq{}}\PY{l+s+s1}{SaleCondition}\PY{l+s+s1}{\PYZsq{}}\PY{p}{]} \PY{o}{==} \PY{l+s+s1}{\PYZsq{}}\PY{l+s+s1}{Partial}\PY{l+s+s1}{\PYZsq{}}\PY{p}{]}\PY{p}{)}\PY{p}{)}
        \PY{n+nb}{print}\PY{p}{(}\PY{n+nb}{len}\PY{p}{(}\PY{n}{df}\PY{p}{[}\PY{n}{df}\PY{p}{[}\PY{l+s+s1}{\PYZsq{}}\PY{l+s+s1}{SaleCondition}\PY{l+s+s1}{\PYZsq{}}\PY{p}{]} \PY{o}{==} \PY{l+s+s1}{\PYZsq{}}\PY{l+s+s1}{AdjLand}\PY{l+s+s1}{\PYZsq{}}\PY{p}{]}\PY{p}{)}\PY{p}{)}
        \PY{n+nb}{print}\PY{p}{(}\PY{n+nb}{len}\PY{p}{(}\PY{n}{df}\PY{p}{[}\PY{n}{df}\PY{p}{[}\PY{l+s+s1}{\PYZsq{}}\PY{l+s+s1}{SaleCondition}\PY{l+s+s1}{\PYZsq{}}\PY{p}{]} \PY{o}{==} \PY{l+s+s1}{\PYZsq{}}\PY{l+s+s1}{Alloca}\PY{l+s+s1}{\PYZsq{}}\PY{p}{]}\PY{p}{)}\PY{p}{)}
        \PY{n+nb}{print}\PY{p}{(}\PY{n+nb}{len}\PY{p}{(}\PY{n}{df}\PY{p}{[}\PY{n}{df}\PY{p}{[}\PY{l+s+s1}{\PYZsq{}}\PY{l+s+s1}{SaleCondition}\PY{l+s+s1}{\PYZsq{}}\PY{p}{]} \PY{o}{==} \PY{l+s+s1}{\PYZsq{}}\PY{l+s+s1}{Family}\PY{l+s+s1}{\PYZsq{}}\PY{p}{]}\PY{p}{)}\PY{p}{)}
\end{Verbatim}


    \begin{Verbatim}[commandchars=\\\{\}]
1198
101
125
4
12
20

    \end{Verbatim}

    \section{Answer 3.4}\label{answer-3.4}

I want to know how many samples of each condition in MSSubClass feature
and Is it has missing values and what is the data type of these column?

So I use loop and unique() and describe to find out

    \begin{Verbatim}[commandchars=\\\{\}]
{\color{incolor}In [{\color{incolor}7}]:} \PY{k}{for} \PY{n}{i} \PY{o+ow}{in} \PY{n}{df}\PY{p}{[}\PY{l+s+s1}{\PYZsq{}}\PY{l+s+s1}{MSSubClass}\PY{l+s+s1}{\PYZsq{}}\PY{p}{]}\PY{o}{.}\PY{n}{unique}\PY{p}{(}\PY{p}{)}\PY{p}{:}
            \PY{n+nb}{print}\PY{p}{(}\PY{n}{i}\PY{p}{,}\PY{n+nb}{len}\PY{p}{(}\PY{n}{df}\PY{p}{[}\PY{n}{df}\PY{p}{[}\PY{l+s+s1}{\PYZsq{}}\PY{l+s+s1}{MSSubClass}\PY{l+s+s1}{\PYZsq{}}\PY{p}{]} \PY{o}{==} \PY{n}{i}\PY{p}{]}\PY{p}{)}\PY{p}{)}
\end{Verbatim}


    \begin{Verbatim}[commandchars=\\\{\}]
60 299
20 536
70 60
50 144
190 30
45 12
90 52
120 87
30 69
85 20
80 58
160 63
75 16
180 10
40 4

    \end{Verbatim}

    \begin{Verbatim}[commandchars=\\\{\}]
{\color{incolor}In [{\color{incolor}8}]:} \PY{n}{df}\PY{p}{[}\PY{l+s+s1}{\PYZsq{}}\PY{l+s+s1}{MSSubClass}\PY{l+s+s1}{\PYZsq{}}\PY{p}{]}\PY{o}{.}\PY{n}{describe}
\end{Verbatim}


\begin{Verbatim}[commandchars=\\\{\}]
{\color{outcolor}Out[{\color{outcolor}8}]:} <bound method NDFrame.describe of 0        60
        1        20
        2        60
        3        70
        4        60
        5        50
        6        20
        7        60
        8        50
        9       190
        10       20
        11       60
        12       20
        13       20
        14       20
        15       45
        16       20
        17       90
        18       20
        19       20
        20       60
        21       45
        22       20
        23      120
        24       20
        25       20
        26       20
        27       20
        28       20
        29       30
               {\ldots} 
        1430     60
        1431    120
        1432     30
        1433     60
        1434     20
        1435     20
        1436     20
        1437     20
        1438     20
        1439     60
        1440     70
        1441    120
        1442     60
        1443     30
        1444     20
        1445     85
        1446     20
        1447     60
        1448     50
        1449    180
        1450     90
        1451     20
        1452    180
        1453     20
        1454     20
        1455     60
        1456     20
        1457     70
        1458     20
        1459     20
        Name: MSSubClass, Length: 1460, dtype: int64>
\end{Verbatim}
            
    \subsection{4. Learning to explore data with
Seaborn}\label{learning-to-explore-data-with-seaborn}

\begin{enumerate}
\def\labelenumi{\arabic{enumi}.}
\tightlist
\item
  Let us first look at the variable we want to predict SalePrice. Use
  Seaborn to plot histogram of sale prices. What do you notice in the
  histogram?
\item
  Plot the histogram of the LotArea variable. What do you notice in the
  histogram?
\item
  Use Seaborn to plot LotArea in the x-axis and SalePrice on the y-axis.
  Try plotting log(LotArea) versus log(SalePrice) and see if the plot
  looks better.
\end{enumerate}

    \section{Answer 4.1}\label{answer-4.1}

The histrogram isn't a Normal Distribution, It has some skewness

    \begin{Verbatim}[commandchars=\\\{\}]
{\color{incolor}In [{\color{incolor}9}]:} \PY{n}{sns}\PY{o}{.}\PY{n}{distplot}\PY{p}{(}\PY{n}{df}\PY{p}{[}\PY{l+s+s1}{\PYZsq{}}\PY{l+s+s1}{SalePrice}\PY{l+s+s1}{\PYZsq{}}\PY{p}{]}\PY{p}{)}
\end{Verbatim}


\begin{Verbatim}[commandchars=\\\{\}]
{\color{outcolor}Out[{\color{outcolor}9}]:} <matplotlib.axes.\_subplots.AxesSubplot at 0x1b8e18cfdd8>
\end{Verbatim}
            
    \begin{center}
    \adjustimage{max size={0.9\linewidth}{0.9\paperheight}}{output_15_1.png}
    \end{center}
    { \hspace*{\fill} \\}
    
    \section{Answer 4.2}\label{answer-4.2}

The histrogram isn't a Normal Distribution, It has very skewness

    \begin{Verbatim}[commandchars=\\\{\}]
{\color{incolor}In [{\color{incolor}10}]:} \PY{n}{sns}\PY{o}{.}\PY{n}{distplot}\PY{p}{(}\PY{n}{df}\PY{p}{[}\PY{l+s+s1}{\PYZsq{}}\PY{l+s+s1}{LotArea}\PY{l+s+s1}{\PYZsq{}}\PY{p}{]}\PY{p}{)}
\end{Verbatim}


\begin{Verbatim}[commandchars=\\\{\}]
{\color{outcolor}Out[{\color{outcolor}10}]:} <matplotlib.axes.\_subplots.AxesSubplot at 0x1b8e92a2ef0>
\end{Verbatim}
            
    \begin{center}
    \adjustimage{max size={0.9\linewidth}{0.9\paperheight}}{output_17_1.png}
    \end{center}
    { \hspace*{\fill} \\}
    
    \section{Answer 4.3}\label{answer-4.3}

The plot after take log in each axises are better because its remove the
skewness of data so the new data are normal distribution

    \begin{Verbatim}[commandchars=\\\{\}]
{\color{incolor}In [{\color{incolor}11}]:} \PY{c+c1}{\PYZsh{}plot before take log}
         \PY{n}{sns}\PY{o}{.}\PY{n}{jointplot}\PY{p}{(}\PY{n}{x} \PY{o}{=} \PY{l+s+s2}{\PYZdq{}}\PY{l+s+s2}{LotArea}\PY{l+s+s2}{\PYZdq{}}\PY{p}{,} \PY{n}{y} \PY{o}{=} \PY{l+s+s2}{\PYZdq{}}\PY{l+s+s2}{SalePrice}\PY{l+s+s2}{\PYZdq{}}\PY{p}{,} \PY{n}{data}\PY{o}{=}\PY{n}{df}\PY{p}{)}
\end{Verbatim}


\begin{Verbatim}[commandchars=\\\{\}]
{\color{outcolor}Out[{\color{outcolor}11}]:} <seaborn.axisgrid.JointGrid at 0x1b8e92d2080>
\end{Verbatim}
            
    \begin{center}
    \adjustimage{max size={0.9\linewidth}{0.9\paperheight}}{output_19_1.png}
    \end{center}
    { \hspace*{\fill} \\}
    
    \begin{Verbatim}[commandchars=\\\{\}]
{\color{incolor}In [{\color{incolor}12}]:} \PY{c+c1}{\PYZsh{}plot after take log}
         \PY{n}{cpdf} \PY{o}{=} \PY{n}{df}\PY{o}{.}\PY{n}{copy}\PY{p}{(}\PY{p}{)} \PY{c+c1}{\PYZsh{} copy dataframe then take log at two column }
         \PY{n}{cpdf}\PY{p}{[}\PY{l+s+s1}{\PYZsq{}}\PY{l+s+s1}{SalePrice}\PY{l+s+s1}{\PYZsq{}}\PY{p}{]} \PY{o}{=} \PY{n}{np}\PY{o}{.}\PY{n}{log}\PY{p}{(}\PY{n}{df}\PY{p}{[}\PY{l+s+s1}{\PYZsq{}}\PY{l+s+s1}{SalePrice}\PY{l+s+s1}{\PYZsq{}}\PY{p}{]}\PY{p}{)}
         \PY{n}{cpdf}\PY{p}{[}\PY{l+s+s1}{\PYZsq{}}\PY{l+s+s1}{LotArea}\PY{l+s+s1}{\PYZsq{}}\PY{p}{]} \PY{o}{=} \PY{n}{np}\PY{o}{.}\PY{n}{log}\PY{p}{(}\PY{n}{df}\PY{p}{[}\PY{l+s+s1}{\PYZsq{}}\PY{l+s+s1}{LotArea}\PY{l+s+s1}{\PYZsq{}}\PY{p}{]}\PY{p}{)}
         
         \PY{n}{sns}\PY{o}{.}\PY{n}{jointplot}\PY{p}{(}\PY{n}{x} \PY{o}{=} \PY{l+s+s2}{\PYZdq{}}\PY{l+s+s2}{LotArea}\PY{l+s+s2}{\PYZdq{}}\PY{p}{,} \PY{n}{y} \PY{o}{=} \PY{l+s+s2}{\PYZdq{}}\PY{l+s+s2}{SalePrice}\PY{l+s+s2}{\PYZdq{}}\PY{p}{,} \PY{n}{data}\PY{o}{=}\PY{n}{cpdf}\PY{p}{)} \PY{c+c1}{\PYZsh{}plot new dataframe}
\end{Verbatim}


\begin{Verbatim}[commandchars=\\\{\}]
{\color{outcolor}Out[{\color{outcolor}12}]:} <seaborn.axisgrid.JointGrid at 0x1b8e96e7c50>
\end{Verbatim}
            
    \begin{center}
    \adjustimage{max size={0.9\linewidth}{0.9\paperheight}}{output_20_1.png}
    \end{center}
    { \hspace*{\fill} \\}
    
    \subsection{5. Dealing with missing
values}\label{dealing-with-missing-values}

\begin{enumerate}
\def\labelenumi{\arabic{enumi}.}
\tightlist
\item
  Suppose we want to start the first step of house price modeling by
  exploring the relationship between four variables: MSSubClass,
  LotArea, LotFrontage and SalePrice. I have done some exploring and
  found out that LotFrontage has a lot of missing values, so you need to
  fix it.
\item
  LotFrontage is the width of the front side of the property. Use Pandas
  to find out how many of the houses in our database is missing
  LotFrontage value.
\item
  Use Pandas to replace NaN values with another number. Since we are
  just exploring and not modeling yet, you can simply replace NaN with
  zeros for now.
\end{enumerate}

    \section{Answer 5}\label{answer-5}

\begin{enumerate}
\def\labelenumi{\arabic{enumi}.}
\setcounter{enumi}{1}
\tightlist
\item
  It has 259 missing values
\item
  use df{[}np.isnan(df{[}'LotFrontage'{]}){]} = 0 to replace NaN by zero
\end{enumerate}

    \begin{Verbatim}[commandchars=\\\{\}]
{\color{incolor}In [{\color{incolor}13}]:} \PY{n+nb}{print}\PY{p}{(}\PY{n+nb}{len}\PY{p}{(}\PY{n}{df}\PY{p}{[}\PY{n}{np}\PY{o}{.}\PY{n}{isnan}\PY{p}{(}\PY{n}{df}\PY{p}{[}\PY{l+s+s1}{\PYZsq{}}\PY{l+s+s1}{LotFrontage}\PY{l+s+s1}{\PYZsq{}}\PY{p}{]}\PY{p}{)}\PY{p}{]}\PY{p}{)}\PY{p}{)} \PY{c+c1}{\PYZsh{}How many NaN in \PYZsq{}LotFrontage\PYZsq{}}
\end{Verbatim}


    \begin{Verbatim}[commandchars=\\\{\}]
259

    \end{Verbatim}

    \begin{Verbatim}[commandchars=\\\{\}]
{\color{incolor}In [{\color{incolor}14}]:} \PY{n}{df}\PY{p}{[}\PY{n}{np}\PY{o}{.}\PY{n}{isnan}\PY{p}{(}\PY{n}{df}\PY{p}{[}\PY{l+s+s1}{\PYZsq{}}\PY{l+s+s1}{LotFrontage}\PY{l+s+s1}{\PYZsq{}}\PY{p}{]}\PY{p}{)}\PY{p}{]} \PY{o}{=} \PY{l+m+mi}{0} \PY{c+c1}{\PYZsh{}replace NaN by zero}
\end{Verbatim}


    \begin{Verbatim}[commandchars=\\\{\}]
{\color{incolor}In [{\color{incolor}15}]:} \PY{n}{df}\PY{p}{[}\PY{l+s+s1}{\PYZsq{}}\PY{l+s+s1}{LotFrontage}\PY{l+s+s1}{\PYZsq{}}\PY{p}{]}\PY{o}{.}\PY{n}{unique}\PY{p}{(}\PY{p}{)} \PY{c+c1}{\PYZsh{}recheck there aren\PYZsq{}t NaN in \PYZsq{}LotFrontage\PYZsq{}}
\end{Verbatim}


\begin{Verbatim}[commandchars=\\\{\}]
{\color{outcolor}Out[{\color{outcolor}15}]:} array([ 65.,  80.,  68.,  60.,  84.,  85.,  75.,   0.,  51.,  50.,  70.,
                 91.,  72.,  66., 101.,  57.,  44., 110.,  98.,  47., 108., 112.,
                 74., 115.,  61.,  48.,  33.,  52., 100.,  24.,  89.,  63.,  76.,
                 81.,  95.,  69.,  21.,  32.,  78., 121., 122.,  40., 105.,  73.,
                 77.,  64.,  94.,  34.,  90.,  55.,  88.,  82.,  71., 120., 107.,
                 92., 134.,  62.,  86., 141.,  97.,  54.,  41.,  79., 174.,  99.,
                 67.,  83.,  43., 103.,  93.,  30., 129., 140.,  35.,  37., 118.,
                 87., 116., 150., 111.,  49.,  96.,  59.,  36.,  56., 102.,  58.,
                 38., 109., 130.,  53., 137.,  45., 106., 104.,  42.,  39., 144.,
                114., 128., 149., 313., 168., 182., 138., 160., 152., 124., 153.,
                 46.])
\end{Verbatim}
            
    \subsection{6. Correlations between multiple
variables}\label{correlations-between-multiple-variables}

One incredible feature of Seaborn is the ability to create correlation
grid with pairplot function. We want to create one single plot that show
us how all variables are correlated. 1. First, you need to create a data
table with four columns: MSSubClass, LotArea (with log function
applied), LotFrontage (missing values replaced) and SalePrice (with log
function applied). 2. Then, use pairplot to create a grid of correlation
plots. What do you observe from this plot?

    \section{Answer 6}\label{answer-6}

\begin{enumerate}
\def\labelenumi{\arabic{enumi}.}
\tightlist
\item
  Create new Dataframe the assign data in each columns
\item
  The plot show that in each subclass has many saleprice and the
  saleprice relative with LotArea and LotFrontage, I found at subclass =
  0 the LotFrontage = 0 Too.
\end{enumerate}

    \begin{Verbatim}[commandchars=\\\{\}]
{\color{incolor}In [{\color{incolor}16}]:} \PY{c+c1}{\PYZsh{}create new data frame}
         \PY{n}{ndf} \PY{o}{=} \PY{n}{pd}\PY{o}{.}\PY{n}{DataFrame}\PY{p}{(}\PY{n}{columns} \PY{o}{=} \PY{p}{[}\PY{l+s+s1}{\PYZsq{}}\PY{l+s+s1}{MSSubClass}\PY{l+s+s1}{\PYZsq{}}\PY{p}{,}\PY{l+s+s1}{\PYZsq{}}\PY{l+s+s1}{LotArea}\PY{l+s+s1}{\PYZsq{}}\PY{p}{,}\PY{l+s+s1}{\PYZsq{}}\PY{l+s+s1}{LotFrontage}\PY{l+s+s1}{\PYZsq{}}\PY{p}{,}\PY{l+s+s1}{\PYZsq{}}\PY{l+s+s1}{SalePrice}\PY{l+s+s1}{\PYZsq{}}\PY{p}{]}\PY{p}{)}
         \PY{n}{ndf}\PY{p}{[}\PY{l+s+s1}{\PYZsq{}}\PY{l+s+s1}{MSSubClass}\PY{l+s+s1}{\PYZsq{}}\PY{p}{]} \PY{o}{=} \PY{n}{df}\PY{p}{[}\PY{l+s+s1}{\PYZsq{}}\PY{l+s+s1}{MSSubClass}\PY{l+s+s1}{\PYZsq{}}\PY{p}{]}
         \PY{n}{ndf}\PY{p}{[}\PY{l+s+s1}{\PYZsq{}}\PY{l+s+s1}{LotArea}\PY{l+s+s1}{\PYZsq{}}\PY{p}{]} \PY{o}{=} \PY{n}{cpdf}\PY{p}{[}\PY{l+s+s1}{\PYZsq{}}\PY{l+s+s1}{LotArea}\PY{l+s+s1}{\PYZsq{}}\PY{p}{]} \PY{c+c1}{\PYZsh{}log function applied}
         \PY{n}{ndf}\PY{p}{[}\PY{l+s+s1}{\PYZsq{}}\PY{l+s+s1}{LotFrontage}\PY{l+s+s1}{\PYZsq{}}\PY{p}{]} \PY{o}{=} \PY{n}{df}\PY{p}{[}\PY{l+s+s1}{\PYZsq{}}\PY{l+s+s1}{LotFrontage}\PY{l+s+s1}{\PYZsq{}}\PY{p}{]} 
         \PY{n}{ndf}\PY{p}{[}\PY{l+s+s1}{\PYZsq{}}\PY{l+s+s1}{SalePrice}\PY{l+s+s1}{\PYZsq{}}\PY{p}{]} \PY{o}{=} \PY{n}{cpdf}\PY{p}{[}\PY{l+s+s1}{\PYZsq{}}\PY{l+s+s1}{SalePrice}\PY{l+s+s1}{\PYZsq{}}\PY{p}{]} \PY{c+c1}{\PYZsh{}log function applied}
\end{Verbatim}


    \begin{Verbatim}[commandchars=\\\{\}]
{\color{incolor}In [{\color{incolor}17}]:} \PY{n}{ndf}\PY{o}{.}\PY{n}{head}\PY{p}{(}\PY{p}{)} \PY{c+c1}{\PYZsh{}Check data frame collect}
\end{Verbatim}


\begin{Verbatim}[commandchars=\\\{\}]
{\color{outcolor}Out[{\color{outcolor}17}]:}    MSSubClass   LotArea  LotFrontage  SalePrice
         0          60  9.041922         65.0  12.247694
         1          20  9.169518         80.0  12.109011
         2          60  9.328123         68.0  12.317167
         3          70  9.164296         60.0  11.849398
         4          60  9.565214         84.0  12.429216
\end{Verbatim}
            
    \begin{Verbatim}[commandchars=\\\{\}]
{\color{incolor}In [{\color{incolor}18}]:} \PY{n}{np}\PY{o}{.}\PY{n}{shape}\PY{p}{(}\PY{n}{ndf}\PY{p}{)} \PY{c+c1}{\PYZsh{}Check new data frame still has 1460 rows }
\end{Verbatim}


\begin{Verbatim}[commandchars=\\\{\}]
{\color{outcolor}Out[{\color{outcolor}18}]:} (1460, 4)
\end{Verbatim}
            
    \begin{Verbatim}[commandchars=\\\{\}]
{\color{incolor}In [{\color{incolor}19}]:} \PY{n}{sns}\PY{o}{.}\PY{n}{pairplot}\PY{p}{(}\PY{n}{ndf}\PY{p}{)}
\end{Verbatim}


\begin{Verbatim}[commandchars=\\\{\}]
{\color{outcolor}Out[{\color{outcolor}19}]:} <seaborn.axisgrid.PairGrid at 0x1b8e92d2a20>
\end{Verbatim}
            
    \begin{center}
    \adjustimage{max size={0.9\linewidth}{0.9\paperheight}}{output_31_1.png}
    \end{center}
    { \hspace*{\fill} \\}
    
    \subsection{7. Data Preparation}\label{data-preparation}

Let's prepare train.csv for model training

\begin{enumerate}
\def\labelenumi{\arabic{enumi}.}
\tightlist
\item
  Pick columns that are numeric data and plot distributions of those
  data (with Seaborn). If you find a column with skewed distribution you
  will write a script to transform that column with a log function. Then
  standardize them.
\item
  For categorical variables, we will simply transform categorical data
  into numeric data by using function \texttt{pandas.get\ dummies()}.
\item
  Split data into x and y. The variable x contains all the house
  features except the SalePrice. y contains only the SalePrice.
\end{enumerate}

    \section{Answer 7}\label{answer-7}

\begin{enumerate}
\def\labelenumi{\arabic{enumi}.}
\tightlist
\item
  Create new dataframe to collect numeric data, remove numeric that be
  categorical data, fill nan value by mean of that column
\item
  Remove skewness by np.log but add 1 in skewed column for resolve -inf
  value after take log(0)
\item
  Standardize by using Z-Score
\item
  Create new dataframe to collect categorical data the one-hot encoding
  by using pd.get\_dummies
\item
  Concatenate numeric data and categorical data in new dataframe
\item
  split to x and y
\end{enumerate}

    \begin{Verbatim}[commandchars=\\\{\}]
{\color{incolor}In [{\color{incolor}20}]:} \PY{n}{df} \PY{o}{=} \PY{n}{pd}\PY{o}{.}\PY{n}{read\PYZus{}csv}\PY{p}{(}\PY{n}{Location}\PY{p}{)} \PY{c+c1}{\PYZsh{}read data}
         \PY{c+c1}{\PYZsh{} choose numerical data}
         \PY{n}{numerical\PYZus{}df} \PY{o}{=} \PY{n}{df}\PY{o}{.}\PY{n}{select\PYZus{}dtypes}\PY{p}{(}\PY{n}{include}\PY{o}{=}\PY{p}{[}\PY{l+s+s1}{\PYZsq{}}\PY{l+s+s1}{number}\PY{l+s+s1}{\PYZsq{}}\PY{p}{,}\PY{l+s+s1}{\PYZsq{}}\PY{l+s+s1}{integer}\PY{l+s+s1}{\PYZsq{}}\PY{p}{,}\PY{l+s+s1}{\PYZsq{}}\PY{l+s+s1}{floating}\PY{l+s+s1}{\PYZsq{}}\PY{p}{]}\PY{p}{)}
         \PY{c+c1}{\PYZsh{} categorical data with numerical data type}
         \PY{n}{cat\PYZus{}numeric} \PY{o}{=} \PY{p}{[}\PY{l+s+s1}{\PYZsq{}}\PY{l+s+s1}{Id}\PY{l+s+s1}{\PYZsq{}}\PY{p}{,}\PY{l+s+s1}{\PYZsq{}}\PY{l+s+s1}{MSSubClass}\PY{l+s+s1}{\PYZsq{}}\PY{p}{,} \PY{l+s+s1}{\PYZsq{}}\PY{l+s+s1}{OverallQual}\PY{l+s+s1}{\PYZsq{}}\PY{p}{,} \PY{l+s+s1}{\PYZsq{}}\PY{l+s+s1}{OverallCond}\PY{l+s+s1}{\PYZsq{}}\PY{p}{,}\PY{l+s+s1}{\PYZsq{}}\PY{l+s+s1}{YearBuilt}\PY{l+s+s1}{\PYZsq{}}\PY{p}{,} \PY{l+s+s1}{\PYZsq{}}\PY{l+s+s1}{YearRemodAdd}\PY{l+s+s1}{\PYZsq{}}\PY{p}{,} \PY{l+s+s1}{\PYZsq{}}\PY{l+s+s1}{GarageYrBlt}\PY{l+s+s1}{\PYZsq{}}\PY{p}{,} \PY{l+s+s1}{\PYZsq{}}\PY{l+s+s1}{MoSold}\PY{l+s+s1}{\PYZsq{}}\PY{p}{,} \PY{l+s+s1}{\PYZsq{}}\PY{l+s+s1}{YrSold}\PY{l+s+s1}{\PYZsq{}}\PY{p}{]}
         \PY{c+c1}{\PYZsh{} drop categorical data}
         \PY{k}{for} \PY{n}{i} \PY{o+ow}{in} \PY{n}{cat\PYZus{}numeric}\PY{p}{:}
             \PY{n}{numerical\PYZus{}df} \PY{o}{=} \PY{n}{numerical\PYZus{}df}\PY{o}{.}\PY{n}{drop}\PY{p}{(}\PY{n}{columns} \PY{o}{=} \PY{n}{i}\PY{p}{)}
         \PY{c+c1}{\PYZsh{} fill NaN Value by mean}
         \PY{n}{numerical\PYZus{}df} \PY{o}{=} \PY{n}{numerical\PYZus{}df}\PY{o}{.}\PY{n}{fillna}\PY{p}{(}\PY{n}{numerical\PYZus{}df}\PY{o}{.}\PY{n}{mean}\PY{p}{(}\PY{p}{)}\PY{p}{)}
         \PY{n}{sns}\PY{o}{.}\PY{n}{pairplot}\PY{p}{(}\PY{n}{numerical\PYZus{}df}\PY{p}{)} \PY{c+c1}{\PYZsh{} plot to see distribution}
\end{Verbatim}


\begin{Verbatim}[commandchars=\\\{\}]
{\color{outcolor}Out[{\color{outcolor}20}]:} <seaborn.axisgrid.PairGrid at 0x1b8eb1d3550>
\end{Verbatim}
            
    \begin{center}
    \adjustimage{max size={0.9\linewidth}{0.9\paperheight}}{output_34_1.png}
    \end{center}
    { \hspace*{\fill} \\}
    
    \begin{Verbatim}[commandchars=\\\{\}]
{\color{incolor}In [{\color{incolor}21}]:} \PY{c+c1}{\PYZsh{}Remove skewness }
         \PY{n}{skew\PYZus{}numerical\PYZus{}df} \PY{o}{=} \PY{n}{numerical\PYZus{}df}\PY{o}{.}\PY{n}{copy}\PY{p}{(}\PY{p}{)} \PY{c+c1}{\PYZsh{}new data frame for after remove skewed }
         \PY{k}{for} \PY{n}{i} \PY{o+ow}{in} \PY{n}{numerical\PYZus{}df}\PY{p}{:}
             \PY{n}{skewness} \PY{o}{=} \PY{n}{numerical\PYZus{}df}\PY{p}{[}\PY{n}{i}\PY{p}{]}\PY{o}{.}\PY{n}{skew}\PY{p}{(}\PY{p}{)}
             \PY{k}{if} \PY{n+nb}{abs}\PY{p}{(}\PY{n}{skewness}\PY{p}{)} \PY{o}{\PYZgt{}}\PY{o}{=} \PY{l+m+mi}{1}\PY{p}{:} \PY{c+c1}{\PYZsh{}skewed distribution if abs(skewness) \PYZgt{}= 1}
                 \PY{n}{skew\PYZus{}numerical\PYZus{}df}\PY{p}{[}\PY{n}{i}\PY{p}{]} \PY{o}{=} \PY{n}{np}\PY{o}{.}\PY{n}{log}\PY{p}{(}\PY{n}{numerical\PYZus{}df}\PY{p}{[}\PY{n}{i}\PY{p}{]}\PY{o}{+}\PY{l+m+mi}{1}\PY{p}{)}
\end{Verbatim}


    \begin{Verbatim}[commandchars=\\\{\}]
{\color{incolor}In [{\color{incolor}22}]:} \PY{c+c1}{\PYZsh{}Standardize by using z\PYZhy{}score}
         \PY{n}{skew\PYZus{}numerical\PYZus{}df} \PY{o}{=} \PY{p}{(}\PY{n}{skew\PYZus{}numerical\PYZus{}df}\PY{o}{\PYZhy{}}\PY{n}{skew\PYZus{}numerical\PYZus{}df}\PY{o}{.}\PY{n}{mean}\PY{p}{(}\PY{p}{)}\PY{p}{)}\PY{o}{/}\PY{n}{skew\PYZus{}numerical\PYZus{}df}\PY{o}{.}\PY{n}{std}\PY{p}{(}\PY{p}{)}
\end{Verbatim}


    \begin{Verbatim}[commandchars=\\\{\}]
{\color{incolor}In [{\color{incolor}23}]:} \PY{c+c1}{\PYZsh{}one\PYZhy{}hot encoding categorical data }
         \PY{n}{cate\PYZus{}df} \PY{o}{=} \PY{n}{df}\PY{o}{.}\PY{n}{select\PYZus{}dtypes}\PY{p}{(}\PY{n}{exclude}\PY{o}{=}\PY{p}{[}\PY{l+s+s1}{\PYZsq{}}\PY{l+s+s1}{number}\PY{l+s+s1}{\PYZsq{}}\PY{p}{,}\PY{l+s+s1}{\PYZsq{}}\PY{l+s+s1}{integer}\PY{l+s+s1}{\PYZsq{}}\PY{p}{,}\PY{l+s+s1}{\PYZsq{}}\PY{l+s+s1}{floating}\PY{l+s+s1}{\PYZsq{}}\PY{p}{]}\PY{p}{)}
         \PY{n}{cate\PYZus{}df} \PY{o}{=} \PY{n}{pd}\PY{o}{.}\PY{n}{get\PYZus{}dummies}\PY{p}{(}\PY{n}{cate\PYZus{}df}\PY{p}{)}
\end{Verbatim}


    \begin{Verbatim}[commandchars=\\\{\}]
{\color{incolor}In [{\color{incolor}24}]:} \PY{n}{cate\PYZus{}num\PYZus{}df} \PY{o}{=} \PY{n}{pd}\PY{o}{.}\PY{n}{DataFrame}\PY{p}{(}\PY{p}{)} \PY{c+c1}{\PYZsh{}new data frame for categorical numerical type}
         \PY{k}{for} \PY{n}{i} \PY{o+ow}{in} \PY{n}{cat\PYZus{}numeric}\PY{p}{:}
             \PY{k}{if} \PY{n}{i} \PY{o}{!=} \PY{l+s+s1}{\PYZsq{}}\PY{l+s+s1}{Id}\PY{l+s+s1}{\PYZsq{}}\PY{p}{:} \PY{c+c1}{\PYZsh{}except \PYZsq{}Id\PYZsq{} }
                 \PY{n}{x} \PY{o}{=} \PY{n}{pd}\PY{o}{.}\PY{n}{get\PYZus{}dummies}\PY{p}{(}\PY{n}{df}\PY{p}{[}\PY{n}{i}\PY{p}{]}\PY{p}{)} \PY{c+c1}{\PYZsh{} one\PYZhy{}hot encoding in each columns}
                 \PY{n}{cate\PYZus{}num\PYZus{}df} \PY{o}{=} \PY{n}{pd}\PY{o}{.}\PY{n}{concat}\PY{p}{(}\PY{p}{[}\PY{n}{cate\PYZus{}num\PYZus{}df}\PY{p}{,}\PY{n}{x}\PY{p}{]}\PY{p}{,}\PY{n}{axis} \PY{o}{=} \PY{l+m+mi}{1}\PY{p}{)} \PY{c+c1}{\PYZsh{}concatenate to data frame in next column}
\end{Verbatim}


    \begin{Verbatim}[commandchars=\\\{\}]
{\color{incolor}In [{\color{incolor}25}]:} \PY{c+c1}{\PYZsh{} Final data frame for already preprocessing data}
         \PY{n}{prepare\PYZus{}df} \PY{o}{=} \PY{n}{pd}\PY{o}{.}\PY{n}{concat}\PY{p}{(}\PY{p}{[}\PY{n}{skew\PYZus{}numerical\PYZus{}df}\PY{p}{,}\PY{n}{cate\PYZus{}num\PYZus{}df}\PY{p}{,}\PY{n}{cate\PYZus{}df}\PY{p}{]}\PY{p}{,}\PY{n}{axis} \PY{o}{=} \PY{l+m+mi}{1}\PY{p}{)}
         \PY{n}{prepare\PYZus{}df}\PY{o}{.}\PY{n}{head}\PY{p}{(}\PY{p}{)}
\end{Verbatim}


\begin{Verbatim}[commandchars=\\\{\}]
{\color{outcolor}Out[{\color{outcolor}25}]:}    LotFrontage   LotArea  MasVnrArea  BsmtFinSF1  BsmtFinSF2  BsmtUnfSF  \textbackslash{}
         0    -0.087176 -0.133225    1.192665    0.779164   -0.355221  -0.944267   
         1     0.563517  0.113374   -0.815680    0.887953   -0.355221  -0.641008   
         2     0.054061  0.419905    1.120646    0.654579   -0.355221  -0.301540   
         3    -0.337486  0.103282   -0.815680    0.384407   -0.355221  -0.061648   
         4     0.716670  0.878130    1.412226    0.754141   -0.355221  -0.174805   
         
            TotalBsmtSF  1stFlrSF  2ndFlrSF  LowQualFinSF          {\ldots}            \textbackslash{}
         0     0.002512 -0.803370  1.161454     -0.133557          {\ldots}             
         1     0.340998  0.418335 -0.794891     -0.133557          {\ldots}             
         2     0.065374 -0.576479  1.188943     -0.133557          {\ldots}             
         3    -0.105783 -0.439270  0.936955     -0.133557          {\ldots}             
         4     0.256149  0.112089  1.617323     -0.133557          {\ldots}             
         
            SaleType\_ConLw  SaleType\_New  SaleType\_Oth  SaleType\_WD  \textbackslash{}
         0               0             0             0            1   
         1               0             0             0            1   
         2               0             0             0            1   
         3               0             0             0            1   
         4               0             0             0            1   
         
            SaleCondition\_Abnorml  SaleCondition\_AdjLand  SaleCondition\_Alloca  \textbackslash{}
         0                      0                      0                     0   
         1                      0                      0                     0   
         2                      0                      0                     0   
         3                      1                      0                     0   
         4                      0                      0                     0   
         
            SaleCondition\_Family  SaleCondition\_Normal  SaleCondition\_Partial  
         0                     0                     1                      0  
         1                     0                     1                      0  
         2                     0                     1                      0  
         3                     0                     0                      0  
         4                     0                     1                      0  
         
         [5 rows x 602 columns]
\end{Verbatim}
            
    \begin{Verbatim}[commandchars=\\\{\}]
{\color{incolor}In [{\color{incolor}26}]:} \PY{c+c1}{\PYZsh{} y collected SalePrice}
         \PY{n}{y} \PY{o}{=} \PY{n}{prepare\PYZus{}df}\PY{o}{.}\PY{n}{loc}\PY{p}{[}\PY{p}{:}\PY{p}{,}\PY{p}{[}\PY{l+s+s1}{\PYZsq{}}\PY{l+s+s1}{SalePrice}\PY{l+s+s1}{\PYZsq{}}\PY{p}{]}\PY{p}{]} 
         
         \PY{c+c1}{\PYZsh{} x collected all except \PYZsq{}SalePrice\PYZsq{}}
         \PY{n}{x} \PY{o}{=} \PY{n}{prepare\PYZus{}df}\PY{o}{.}\PY{n}{loc}\PY{p}{[}\PY{p}{:}\PY{p}{,}\PY{p}{:}\PY{p}{]} 
         \PY{n}{x} \PY{o}{=} \PY{n}{x}\PY{o}{.}\PY{n}{drop}\PY{p}{(}\PY{n}{columns} \PY{o}{=} \PY{l+s+s1}{\PYZsq{}}\PY{l+s+s1}{SalePrice}\PY{l+s+s1}{\PYZsq{}}\PY{p}{)}
\end{Verbatim}


    \subsection{8. Let us first fit a very simple linear regression model,
just to see what we
get.}\label{let-us-first-fit-a-very-simple-linear-regression-model-just-to-see-what-we-get.}

\begin{enumerate}
\def\labelenumi{\arabic{enumi}.}
\tightlist
\item
  Use import LinearRegression from sklearn.linear model and use function
  \texttt{fit()} to fit the model.
\item
  Use function \texttt{predict()} to get house price predictions from
  the model (let's call the predicted house prices yhat).
\item
  Plot \texttt{y} against \texttt{yhat} to see how good your predictions
  are.
\end{enumerate}

    \section{Answer 8}\label{answer-8}

กราฟที่ได้จากการ plot ดูแล้วมีแนวโน้มเป็นเส้นตรง แสดงว่าการ model
predict ได้ค่อนข้างดี

    \begin{Verbatim}[commandchars=\\\{\}]
{\color{incolor}In [{\color{incolor}27}]:} \PY{n}{regr} \PY{o}{=} \PY{n}{linear\PYZus{}model}\PY{o}{.}\PY{n}{LinearRegression}\PY{p}{(}\PY{p}{)} \PY{c+c1}{\PYZsh{} create regresion model}
         \PY{n}{regr}\PY{o}{.}\PY{n}{fit}\PY{p}{(}\PY{n}{x}\PY{p}{,}\PY{n}{y}\PY{p}{)} \PY{c+c1}{\PYZsh{}fit model}
         \PY{n}{yhat} \PY{o}{=} \PY{n}{regr}\PY{o}{.}\PY{n}{predict}\PY{p}{(}\PY{n}{x}\PY{p}{)} \PY{c+c1}{\PYZsh{}use regression model to predict }
         \PY{c+c1}{\PYZsh{} plot y againt}
         \PY{n}{plt}\PY{o}{.}\PY{n}{plot}\PY{p}{(}\PY{n}{y}\PY{p}{,}\PY{n}{yhat}\PY{p}{,}\PY{l+s+s1}{\PYZsq{}}\PY{l+s+s1}{r.}\PY{l+s+s1}{\PYZsq{}}\PY{p}{)} 
         \PY{n}{plt}\PY{o}{.}\PY{n}{ylabel}\PY{p}{(}\PY{l+s+s1}{\PYZsq{}}\PY{l+s+s1}{yhat}\PY{l+s+s1}{\PYZsq{}}\PY{p}{)}
         \PY{n}{plt}\PY{o}{.}\PY{n}{xlabel}\PY{p}{(}\PY{l+s+s1}{\PYZsq{}}\PY{l+s+s1}{y}\PY{l+s+s1}{\PYZsq{}}\PY{p}{)}
\end{Verbatim}


\begin{Verbatim}[commandchars=\\\{\}]
{\color{outcolor}Out[{\color{outcolor}27}]:} Text(0.5,0,'y')
\end{Verbatim}
            
    \begin{center}
    \adjustimage{max size={0.9\linewidth}{0.9\paperheight}}{output_43_1.png}
    \end{center}
    { \hspace*{\fill} \\}
    
    \subsection{9. Assessing Your Model}\label{assessing-your-model}

According to Kaggle's official rule on this problem, they use root mean
square errors (rmse) to judge the accuracy of our model. This error
computes the dif- ference between the log of actual house prices and the
log of predicted house price. Find the mean and squareroot them.

We want to see how we compare to other machine learning contestants on
Kag- gle so let us compute our rmse. Luckily, sklearn has done most of
the work for you by providing mean square error function. You can use it
by importing the function from sklearn.metrics. Then, you can compute
mean square error and take a squareroot to get rmse.

What's the rmse of your current model? Check out Kaggle Leaderboard for
this problem to see how your number measures up with the other
contestants.

    \section{Answer 9}\label{answer-9}

the root means square error is 0.195627

    \begin{Verbatim}[commandchars=\\\{\}]
{\color{incolor}In [{\color{incolor}28}]:} \PY{n}{rms} \PY{o}{=} \PY{n}{np}\PY{o}{.}\PY{n}{sqrt}\PY{p}{(}\PY{n}{np}\PY{o}{.}\PY{n}{square}\PY{p}{(}\PY{n}{y}\PY{o}{\PYZhy{}}\PY{n}{yhat}\PY{p}{)}\PY{o}{.}\PY{n}{mean}\PY{p}{(}\PY{p}{)}\PY{p}{)}
         \PY{n+nb}{print}\PY{p}{(}\PY{n}{rms}\PY{p}{)}
\end{Verbatim}


    \begin{Verbatim}[commandchars=\\\{\}]
SalePrice    0.195627
dtype: float64

    \end{Verbatim}

    \subsection{10. Cross Validation}\label{cross-validation}

As we discussed earlier, don't brag about your model's accuracy until
you have performed cross validation. Let us check cross-validated
performance to avoid embarrassment.

Luckily, scikit learn has done most of the work for us once again. You
can use the function \texttt{cross\_val\_predict()} to train the model
with cross validation method and output the predictions.

What's the rmse of your cross-validated model? Discuss what you observe
in your results here. You may try plotting this new yhat with y to get
better insights about this question.

    \section{Answer10}\label{answer10}

จากกราฟที่ได้ พบว่า model predict ได้แย่มากถ้าใช้ cross validation
method เมื่อเทียบกับ predict แบบธรรมดา โดย rms อันใหม่มีค่าเท่ากับ
2.590273e+09 หมายความว่า model predict error เยอะมาก

    \begin{Verbatim}[commandchars=\\\{\}]
{\color{incolor}In [{\color{incolor}29}]:} \PY{n}{cross\PYZus{}val\PYZus{}y} \PY{o}{=} \PY{n}{cross\PYZus{}val\PYZus{}predict}\PY{p}{(}\PY{n}{regr}\PY{p}{,}\PY{n}{x}\PY{p}{,}\PY{n}{y}\PY{p}{)}
\end{Verbatim}


    \begin{Verbatim}[commandchars=\\\{\}]
{\color{incolor}In [{\color{incolor}30}]:} \PY{n}{rms\PYZus{}cross} \PY{o}{=} \PY{n}{np}\PY{o}{.}\PY{n}{sqrt}\PY{p}{(}\PY{n}{np}\PY{o}{.}\PY{n}{square}\PY{p}{(}\PY{n}{y}\PY{o}{\PYZhy{}}\PY{n}{cross\PYZus{}val\PYZus{}y}\PY{p}{)}\PY{o}{.}\PY{n}{mean}\PY{p}{(}\PY{p}{)}\PY{p}{)}
         \PY{n}{rms\PYZus{}cross}
\end{Verbatim}


\begin{Verbatim}[commandchars=\\\{\}]
{\color{outcolor}Out[{\color{outcolor}30}]:} SalePrice    2.590273e+09
         dtype: float64
\end{Verbatim}
            
    \begin{Verbatim}[commandchars=\\\{\}]
{\color{incolor}In [{\color{incolor}31}]:} \PY{n}{plt}\PY{o}{.}\PY{n}{plot}\PY{p}{(}\PY{n}{y}\PY{p}{,}\PY{n}{cross\PYZus{}val\PYZus{}y}\PY{p}{,}\PY{l+s+s1}{\PYZsq{}}\PY{l+s+s1}{r.}\PY{l+s+s1}{\PYZsq{}}\PY{p}{)}
         \PY{n}{plt}\PY{o}{.}\PY{n}{ylabel}\PY{p}{(}\PY{l+s+s1}{\PYZsq{}}\PY{l+s+s1}{cross\PYZus{}val\PYZus{}y}\PY{l+s+s1}{\PYZsq{}}\PY{p}{)}
         \PY{n}{plt}\PY{o}{.}\PY{n}{xlabel}\PY{p}{(}\PY{l+s+s1}{\PYZsq{}}\PY{l+s+s1}{y}\PY{l+s+s1}{\PYZsq{}}\PY{p}{)}
\end{Verbatim}


\begin{Verbatim}[commandchars=\\\{\}]
{\color{outcolor}Out[{\color{outcolor}31}]:} Text(0.5,0,'y')
\end{Verbatim}
            
    \begin{center}
    \adjustimage{max size={0.9\linewidth}{0.9\paperheight}}{output_51_1.png}
    \end{center}
    { \hspace*{\fill} \\}
    
    \subsection{11 (Optional) Fit Better
Models}\label{optional-fit-better-models}

There are other models you can fit that will perform better than linear
regres- sion. For example, you can fit linear regression with L2
regularization. This class of models has a street name of `Ridge
Regression' and sklearn simply called them Ridge. As we learned last
time, this model will fight overfitting problem. Furthermore, you can
try linear regression with L1 regularization (street name Lasso
Regression or Lasso in sklearn). Try these models and see how you com-
pare with other Kagglers now. You can write about your findings below.

    \begin{Verbatim}[commandchars=\\\{\}]
{\color{incolor}In [{\color{incolor}32}]:} \PY{c+c1}{\PYZsh{} create Lasso and Ridge regression model }
         \PY{n}{lsso} \PY{o}{=} \PY{n}{linear\PYZus{}model}\PY{o}{.}\PY{n}{Lasso}\PY{p}{(}\PY{p}{)}
         \PY{n}{ridge} \PY{o}{=} \PY{n}{linear\PYZus{}model}\PY{o}{.}\PY{n}{Ridge}\PY{p}{(}\PY{p}{)}
         
         \PY{c+c1}{\PYZsh{}fit model}
         \PY{n}{lsso}\PY{o}{.}\PY{n}{fit}\PY{p}{(}\PY{n}{x}\PY{p}{,}\PY{n}{y}\PY{p}{)}
         \PY{n}{ridge}\PY{o}{.}\PY{n}{fit}\PY{p}{(}\PY{n}{x}\PY{p}{,}\PY{n}{y}\PY{p}{)}
\end{Verbatim}


\begin{Verbatim}[commandchars=\\\{\}]
{\color{outcolor}Out[{\color{outcolor}32}]:} Ridge(alpha=1.0, copy\_X=True, fit\_intercept=True, max\_iter=None,
            normalize=False, random\_state=None, solver='auto', tol=0.001)
\end{Verbatim}
            
    \begin{Verbatim}[commandchars=\\\{\}]
{\color{incolor}In [{\color{incolor}33}]:} \PY{c+c1}{\PYZsh{}predict data}
         \PY{n}{lsso\PYZus{}y} \PY{o}{=} \PY{n}{lsso}\PY{o}{.}\PY{n}{predict}\PY{p}{(}\PY{n}{x}\PY{p}{)}
         \PY{n}{ridge\PYZus{}y} \PY{o}{=} \PY{n}{ridge}\PY{o}{.}\PY{n}{predict}\PY{p}{(}\PY{n}{x}\PY{p}{)}
\end{Verbatim}


    \begin{Verbatim}[commandchars=\\\{\}]
{\color{incolor}In [{\color{incolor}34}]:} \PY{n}{lsso\PYZus{}y} \PY{o}{=} \PY{n}{np}\PY{o}{.}\PY{n}{reshape}\PY{p}{(}\PY{n}{lsso\PYZus{}y}\PY{p}{,}\PY{p}{(}\PY{l+m+mi}{1460}\PY{p}{,}\PY{l+m+mi}{1}\PY{p}{)}\PY{p}{)} \PY{c+c1}{\PYZsh{}shape of lsso\PYZus{}y is (1460,) So I reshape it to (1460,1)}
\end{Verbatim}


    \begin{Verbatim}[commandchars=\\\{\}]
{\color{incolor}In [{\color{incolor}35}]:} \PY{c+c1}{\PYZsh{}calculate root mean square error}
         \PY{n}{rms\PYZus{}lsso} \PY{o}{=} \PY{n}{np}\PY{o}{.}\PY{n}{sqrt}\PY{p}{(}\PY{n}{np}\PY{o}{.}\PY{n}{square}\PY{p}{(}\PY{n}{y}\PY{o}{\PYZhy{}}\PY{n}{lsso\PYZus{}y}\PY{p}{)}\PY{o}{.}\PY{n}{mean}\PY{p}{(}\PY{p}{)}\PY{p}{)}
         \PY{n}{rms\PYZus{}ridge} \PY{o}{=} \PY{n}{np}\PY{o}{.}\PY{n}{sqrt}\PY{p}{(}\PY{n}{np}\PY{o}{.}\PY{n}{square}\PY{p}{(}\PY{n}{y}\PY{o}{\PYZhy{}}\PY{n}{ridge\PYZus{}y}\PY{p}{)}\PY{o}{.}\PY{n}{mean}\PY{p}{(}\PY{p}{)}\PY{p}{)}
         \PY{n+nb}{print}\PY{p}{(}\PY{n}{rms\PYZus{}lsso}\PY{p}{)}
         \PY{n+nb}{print}\PY{p}{(}\PY{n}{rms\PYZus{}ridge}\PY{p}{)}
\end{Verbatim}


    \begin{Verbatim}[commandchars=\\\{\}]
SalePrice    0.999657
dtype: float64
SalePrice    0.212888
dtype: float64

    \end{Verbatim}


    % Add a bibliography block to the postdoc
    
    
    
    \end{document}
